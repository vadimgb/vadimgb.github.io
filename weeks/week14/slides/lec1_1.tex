\begin{comment}
На основе имеющихся аксиом люди могут доказать истинность теорем.
	Эти исходные утверждения(аксиомы) мы будем назвать базой знаний.
\end{comment}

\begin{frame}
	\frametitle{Агент, использующий знания.}
	Агент, использующий знания --  это агент, принимающий решение, основываясь на 
	внутреннем представлении знаний.
\end{frame}
\begin{frame}
	\begin{itemize}[label={}]
		\item Если не будет дождя, то Гарри посетит Хагрида.
		\item Гарри сегодня посетил Хагрида или Дамблдора, но не обоих.
		\item Гарри посетил Дамблдора сегодня.
		\pause
		\item \textbf{Гарри сегодня не посещал Хагрида.}
		\pause
		\item \textbf{Сегодня был дождь.}
	\end{itemize}
\end{frame}
\begin{frame}
	\begin{center}
		\Huge Логика
	\end{center}
\end{frame}
\begin{frame}
	\Large \textbf{Предложение} - суждение о мире на языке
	представления знаний.
\end{frame}

\begin{comment}
	Язык представления знаний - способ представления знания.
	 Один из таких языков - логика высказываний.
\end{comment}

\begin{frame}
	\begin{center}
	\Huge Логика высказываний \large(пропозициональная)
	\end{center}
\end{frame}

\begin{comment}
	Символы представляют высказывания об мире.  
\end{comment}

\begin{frame}
	\begin{center}
	\Huge Пропозициональная символы

	P\; Q\; R
	\end{center}
\end{frame}

\begin{frame}
	\frametitle{Знаки логических связок}
\Large
	\begin{center}
		\begin{table}
			\begin{tabular}{c  c  c}
				$¬$  &  $∧$ & $∨$ \\
				отрицание & конъюнкция & дизъюнкция\\
			\end{tabular}

			\begin{tabular}{c c}
				$\Rightarrow$ & $\Leftrightarrow$\\
				импликация & эквиваленция
			\end{tabular}
		\end{table}
	\end{center}
\end{frame}
\begin{frame}
	\Large
	\frametitle{Отрицание $¬$}
	\begin{center}
		\begin{tabular}{|c | c|}
		\hline
			\rowcolor{darkBlue}
		$P$ & $¬ P$ \\ \hline
		False & True \\ \hline
		True & False \\ \hline
	\end{tabular}
	\end{center}
\end{frame}

\begin{frame}
	\Large
	\frametitle{Конъюнкция, и, $∧$}
	\begin{center}
	\begin{table}
		\begin{tabular}{|c|c|c|}
			\hline
			\rowcolor{darkBlue}
			$P$ & 		$Q$ &		 $P ∧ Q$ \\ \hline
			False & 	False & 	False \\ \hline
			False & 	True & 		False \\ \hline
			True  & 	False & 	Fasle \\ \hline
			True & 		True & 		True \\ \hline
		\end{tabular}
	\end{table}
	\end{center}
\end{frame}

\begin{frame}
	\Large
	\frametitle{Дизъюнкция, или, $∨$}
	\begin{center}
	\begin{table}
		\begin{tabular}{|c|c|c|}
			\hline
			\rowcolor{darkBlue}
			$P$ & 		$Q$ &		 $P ∨ Q$ \\ \hline
			False & 	False & 	False \\ \hline
			False & 	True & 		True \\ \hline
			True  & 	False & 	True \\ \hline
			True & 		True & 		True \\ \hline
		\end{tabular}
	\end{table}
	\end{center}
\end{frame}

\begin{frame}
	\Large
	\frametitle{Импликация $\Rightarrow$}
	\begin{center}
	\begin{table}
		\begin{tabular}{|c|c|c|}
			\hline
			\rowcolor{darkBlue}
			$P$ & 		$Q$ &		 $P \Rightarrow Q$ \\ \hline
			False & 	False & 	True \\ \hline
			False & 	True & 		True \\ \hline
			True  & 	False & 	False \\ \hline
			True & 		True & 		True \\ \hline
		\end{tabular}
	\end{table}
	\end{center}
\end{frame}

\begin{frame}
	\Large
	\frametitle{Эквиваленция $\Leftrightarrow$}
	\begin{center}
		\begin{tabular}{|c|c|c|}
			\hline
			\rowcolor{darkBlue}
			$P$ & 		$Q$ &		 $P \Leftrightarrow Q$ \\ \hline
			False & 	False & 	True \\ \hline
			False & 	True & 		False \\ \hline
			True  & 	False & 	False \\ \hline
			True & 		True & 		True \\ \hline
		\end{tabular}
	\end{center}
\end{frame}


